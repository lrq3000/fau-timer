\title{FauTimerReport}
\author{
        Sebastian Schinzel
            \and
        Isabell Schmitt
}
\date{\today}

\documentclass[12pt]{article}

\usepackage{amssymb}
\usepackage{nth}

\usepackage{color}
\usepackage{marginnote}
\newcommand{\id}[1]{\mbox{\textit{#1}}}
\newcommand{\todo}[1]{\textcolor{red}{\marginnote{TODO} To Do: #1}}

\begin{document}
\maketitle

% %\begin{abstract}
% %This is the paper's abstract \ldots
% %\end{abstract}
% %
% %\section{Introduction}
% %A timing attack gets increasingly more dangerous the fewer measurements the attacker
% %has to perform. Crosby, Wallach, and Riedi \cite{bib:crosby} studied different 
% %hypothesis tests and their suitability for analysing differences timing measurements.
% %They proposed and informally described the \emph{box test}, which performed best 
% %in their analysis.
% %
% %In the following, we formalize the box test and propose an algorithm that finds the minimum amount
% %of timing measurements required for the box test to still find differences. This
% %is important information for the attacker because the fewer measurements are required,
% %the smaller the risk for the attacker that the attack is either detected or prevented.
% %
% %\todo{Mention that exact timing measurements are difficult because of noise}
% %
% %\section{Model}
% %\subsection{Definitions}
% %A timing side channel appears if the response time $t\in T$ of a procedure depends on a secret
% %value $s\in S$ that the procedure accesses. As the 
% %
% %% Let $a\in A$ and $b\in B$ be two measurements of the form $(n,t,s)$, where $n\in \mathbb{N}$ is
% %% the $n^{th}$ measurement of the timing $t\in T$. The timing was measured 
% %% using a secret $s\in S$.
% %
% %A percentile $p\in P: 0 \leq p < 100$ denotes a single timing $t$ that is referenced by its rank
% %in the sorted measurement $m$ by $n=\frac{P|m|}{100}+\frac{1}{2}$, where $|m|$ denotes the number
% %of timings in $m$. Take the measurement $m_1=\langle 0,1,2,3,4,5,6,7,8,9 \rangle$ as an example. The
% %$\nth{15}$ percentile ($p=15$) is $\frac{15*10}{100}+\frac{1}{2}=2$ and the $\nth{2}$ rank in $m_1$
% %is $1$. The $\nth{65}$ percentile is denoted by the $\frac{65*10}{100}+\frac{1}{2}=\nth{7}$ rank in
% %$m_1$, which is $6$.
% %
% %A box is a function $b(p_i, p_j, m) \to m_b$ where $i,j\in \mathbb{N}: 0 \leq i < j < |m|$, $p_i$ is the lower percentile and
% %$p_j$ is the upper percentile of the box. The box returns the subset of measurements $m_b \subseteq m$
% %that is between the lower percentile and the upper percentile.
% %For example, $b(15,65,\langle 0,1,2,3,4,5,6,7,8,9 \rangle)$ results in the subset
% %$b=\langle 2,3,4,5,6 \rangle$. 
% %
% %A box test $\id{bt}(m_1, m_2, p_i, p_j) \to {\texttt{true}|\texttt{false}}$ takes two different timing measurements $m_1$, $m_2$ 
% %and the box $b$ as parameter. It returns $\texttt{true}$ if $m_{b1}=b(p_i, p_j, m_1$ and $m_{b2}=b(p_i, p_j, m_2)$ do not
% %overlap, i.e.\ $m_{b1}\cap m_{b2}=\emptyset$. 
% %
% %\subsection{Box Test Calibration}
% %The box test takes the lower and the upper percentiles ($p_i$, $p_j$) as parameter
% %and in this section, we describe an algorithm that searches for the ideal parameters.
% %Furthermore, 
% %\pagebreak
% %
% %
\section{FAUTimer Algorithm}
\subsection{Definitions}
 
\begin{itemize}
  \item A measurement set $A$ is a set of $n$ tuples $A\in M: A=\langle(i_1,s,t_1), (i_2,s,t_2), \ldots, (n,s,t_n)\rangle$ where
    $i\in \mathbb{N}: 1 \leq i \leq n$ is a unique index, $t\in T$ is a timing measurement and $s\in S$ is a secret value
    under which $t$ was measured. In the following, we assume that the there are two different 
    measurement sets
    $A,B\in M: A=\langle (1,s_1,t_1), (2,s_1,t_3), \ldots (n,s_1,t_{2n-1})\rangle \land B=\langle(1,s_2,t_2), (2,s_2,t_4), \ldots, (n,s_2,t_{2n})\rangle
    \land s_1\neq s_2$ that are sorted by the index in ascending order. Furthermore, the notation $\hat{A}$ denotes a measurement set
    $A$ that is sorted by $t$ in ascending order.

  \item A subset of the measurement $A_{u,v}\in A$ starts at position $u$ in the measurement set $A$, which
  is sorted by the index, and ends at position $v$. For example, let 
  $A=\langle (1,s_1,t_1), (2,s_1,t_2), \ldots (100,s_1,t_{100})\rangle$ and $|A|=100$, then 
  $A_{50,100}=\langle (50,s_1,t_{50}), (51,s_1,t_{51}), \ldots (100,s_1,t_{100})\rangle$ and $|A_{50,100}|=50$.

  \item A \emph{percentile} is a function $p(\hat{A}, q)\to t$ (abbreviated $\hat{A}_{[q]} \to t$) that takes a measurement
    set $\hat{A}$ that is sorted by the timing values in ascending order and an index
    $q\in \mathbb{N}: 0 \leq q \leq 100$. It returns a single timing measurement $t$ whose
    actual rank $x$ is determined by
    \begin{equation}
    x=\left\{
      \begin{array}{rl} 
        \frac{i}{100}|\hat{A}| & \id{ if } q < 100\\
             |\hat{A}| -1      & \id{ if } q = 100\\
      \end{array} \right.
    \end{equation}
    As an example, let
    $\hat{A}=\langle (1,s,2), (2,s,4), (3,s,6), \ldots, (1000,s,2000)\rangle $ and $|\hat{A}|=1000$, then $\hat{A}_{[0]}=2$, $\hat{A}_{[57]}=114$, $\hat{A}_{[99]}=1980$, and $\hat{A}_{[100]}=2000$.

  \item A \emph{box} is a function $b(\hat{A}_{p_l,p_h}) \to \hat{A}_{[p_l,p_h]}$ that takes a sorted measurement set $\hat{A}\in M$ and the
    two percentile indices
    $p_l,p_h\in \mathbb{N}: 0 \leq p_l < p_h \leq 100$ as parameters. It returns a measurement set $\hat{A}_{[p_l,p_h]}\subseteq A$
    which contains all tuples $(p_l,s,t): \hat{A}_{[p_l]}\leq t \leq \hat{A}_{[p_h]}$.\\
    If you take the former $\hat{A}$ as an example,
    then $\hat{A}_{[5,8]}=\langle (5,s,10),(6,s,12),(7,s,14),(8,s,18) \rangle$.

  \item A \emph{box test} is a function $\id{bt}(\hat{A}, \hat{B}, p_l, p_h) \to \{\id{true}|\id{false}\}$ that takes two
    sorted measurement sets $\hat{A},\hat{B}$ and two percentile indices and returns whether the timing
    measurements in $\hat{A},\hat{B}$ are significantly different (\id{true}), or not (\id{false}).
    The test works as follows:
    \begin{enumerate}
        \item apply the box to both measurement sets, i.e.\ $b(\hat{A},p_l,p_h)=\hat{A}_{[p_l,p_h]}$ and $b(\hat{B},p_l,p_h)=\hat{B}_{[p_l,p_h]}$.
        \item check whether the timings in both subsets overlap, i.e. $\hat{A}_{[p_l]} < \hat{B}_{[p_h]}$ and $\hat{B}_{[p_l]} < \hat{A}_{[p_h]}$.
        \item if they overlap, then no significant timing differences were detected, and the box test returns \id{false}.
        \item if they do not overlap, then a significant timing differences was detected, and the box test returns \id{true}.
    \end{enumerate}

 \end{itemize}


\subsection{Calibration Phase}
The calibration algorithm returns two results:
\begin{itemize}
  \item the two percentile indices of the optimal box ($p_l$, $p_h$),
  \item the approximate minimum amount of required measurements for the box test to return significant differences.
\end{itemize}

\begin{enumerate}
\item Read the timing measurements and split them according to the secret. We assume here that there are two secrets and thus two measurement sets $A,B\in M$.

\item We assume in the following that $\hat{A}, \hat{B}$ are sorted by the timings in ascending order.

\item Approximate minimum amount of required timing measurements:\\
  Loop over all possible boxes with $p_l,p_h\in \mathbb{N}: 0 \leq p_l < p_h \leq 100$,
  example:\\ $(0,1), (0,2), \ldots (0,100), (1,2), (1,3), \ldots, (1,100), (2,3), \ldots, (99,100)$:
  \begin{enumerate}
    \item perform the box test, i.e. $\id{bt}(\hat{A}, \hat{B}, [p_l], [p_h])$.
    \item If the box test returns \id{false}, continue in the
      loop with the next box. If the loop is finished, go to step 4.
    \item If the box test returns \id{true}, perform the following steps:
      \begin{itemize}
        \item \textcolor{red}{discard the residual boxes in this iteration of the loop.}
        \item Remove the first half of the measurements in $A$ and $B$, resulting in $A_{50,100}, B_{50,100}$.
        \item Goto step 3 using the bisected sets $A_{50,100}, B_{50,100}$.
      \end{itemize}
   \end{enumerate}

\item Find optimal parameters for box test:
  In the former steps, we approximated the minimum amount of measurements.
  Did the algorithm ever enter step \textcolor{red}{3(d)}, i.e. was the amount of measurements
  reduced at least once?
  \begin{enumerate}
    \item If yes, then we just slipped below the minimum amount of required measurements.
    Now undo step \textcolor{red}{3(d)i} (double the size of $A,B$, i.e. use $A_{50,100}$ instead of
    $A_{75,100}$. In the next step, we find the optimal box test parameters. 
    Loop over all possible boxes with $p_l,p_h\in \mathbb{N}: 0 \leq p_l < p_h \leq 100$,
    example:\\ $(0,1), (0,2), \ldots (0,100), (1,2), (1,3), \ldots, (1,100), (2,3), \ldots, (99,100)$:

      \begin{enumerate}
        \item perform the box test, i.e. $\id{bt}(\hat{A}, \hat{B}, [p_l], [p_h])$.
        \item If the box test returns \id{false}, continue in the
          loop with the next box. If the loop is finished, go to step 5.
        \item If the box test returns \id{true}, remember the parameters ($p_l,p_h$) of the box and the
          ``size'' of the box ($p_l-p_h$). Only remember the biggest box and return this to the user after
          all box test parameters were tested. Afterwards, the user knows the minimum amount of measurements
          and the optimal box test parameters.
        \end{enumerate}
    \item If no, return that the timing measurements $A,B$ have no significant differences. Tell
    user that he might need to perform more measurements.
  \end{enumerate}


\item Validation of calibration phase:\\
  In step 3, 4, and 5, we recursively bisected the sets $A,B$ to find the approximate minimum amount of
  measurements and the optimal box parameters. We now validate these results on the residual measurements. 
  \begin{enumerate}
    \item Split $A,B$ in a way such that each subset has the minimum amount of measurements. For example, let
      $|A|=1000$ and the minimum amount of measurements be $50$. Then $A$ can be split in 20
      subsets $A_1, A_2, \ldots, A_20$. There $n$ subsets, each with 
      $$n=\frac{|A|}{\texttt{minimum amount of measurements}}$$.
      The residual measurements are not used in this step. As the first measurements of a datasets often
      are more noisy than the following measurements, the residual measurements should be the first 
      measurements.
    \item run the box test on $A_n,B_n$ and verify that the optimal box in all subsets do NOT overlap.
      If it overlaps, double the size of $A,B$ and go back to step 4(a).
    \item Run the box test on $A_n,A_n$ and verify that the optimal box in all subsets DO overlap
      If it does not overlap, double the size of $A,B$ and go back to step 4(a).
    \item Run the box test on $B_n,B_n$ and verify that the optimal box in all subsets DO overlap
      If it does not overlap, double the size of $A,B$ and go back to step 4(a).
  \end{enumerate}


\item Reporting: Print out the following information (example values):
  \begin{enumerate}
    \item The usual PDF report
    \item On stdout with some ASCII art for $s_1$\textless $s_2$ and $s_2$\textless $s_1$: 
    \begin{enumerate}
    	\item Approximate minimum amount of required measurements: 123
    	\item i for optimal box: 5
    	\item j for optimal box: 11
    \end{enumerate}
  \end{enumerate}
\end{enumerate}


% \bibliographystyle{abbrv}
% \bibliography{main}

\end{document}
